%TUTAJ JEST MAŁA DOKUMENTACJA WRAZ Z PYTANIAMI
\documentclass[9pt]{article}
\usepackage[utf8]{inputenc}
\usepackage[MeX]{polski}
\usepackage{amsfonts}
\newcommand{\Q}{\mathbb{Q}}
\newcommand{\HS}{\mathcal{H}}
\newcommand{\V}{\mathcal{V}}
\newcommand{\C}{\mathcal{C}}
\newcommand{\IC}{ \mathcal{I}_{\Q^M}^{\C}}
\newcommand{\IV}{ \mathcal{I}_{\Q^N}^{\Q^M}}

\begin{document}
%SCHEMAT PRACY KLASY
\section{Schemat działania}

%DEFINICJIE MATEMATYCZNE
\section{Model}
\subsection{Obiekty}
Podstawowe definicje zaczniemy od przestrzeni rozwiązań i matematycznychobietków modlowych.
\begin{itemize}
\item $\Q$ - zbiór liczb, nad którym liczymy.
\item $X_N \subset X, X_N \simeq \{1,2 ...N\}$ - przestrzeń punktów 
w których znamy potencjał.
\item $X_M \subset X$ - punkty gdzie umieszczono źródła.
\item $X =M>N$ - pełna przestrzeń dla potencjałów.
\item $\HS$ - prawdziwa przestrzeń funkcji, których aproksymacji szukamy
    tzn. są tutaj zarówno rozwiązania (prądy) jaki i potencjały (w teorii).
\item $\C = lin(\{ c_x: x \in X_M, c_x(t) \sim gauss(t-x), t\in X \}) \subset \HS$ 
- przestrzeń funkcyjna skończenie wymiarowa do aproksymacji prądów.
\item $\V \simeq {\Q^M}, \V \subset \HS \V=L\C$ - przestrzeń do aproksymacji potencjałów.
\item $L: \C \mapsto \V$ - liniowy operator, liczy potencjały z prądów
z grubsza odwrócony laplasjan.
\item $K \simeq \mathcal{K}|_{X_N \times  X_N}, K: \Q^N \mapsto \Q^N$ - jądro 
reproduktywne przestrzeni aproksymacyjnej $V$, tj. $\mathcal{K}$
w obcięciu do jej pewnej podprzestrzni.
\item $\IC: \Q^M \mapsto \C$ macierz zmiany reoprezentacji z przestrzeni
aporoksymacyjnej w potenjałach do przestrzeni funkcji.
\item $\IV: \Q^N \mapsto \Q^M$ - macierz która oblicza wartości funkcji bazowych 
potencjału w punktach zadanych przez pomiar.
\end{itemize}
Przydatne równości:
$K({x,y}) = (k_y| k_x) = k_x(y)$


\subsection{Obiekty}
Niżej wypisano obiekty wraz z matematycznymi strukturami które implementują.
Strzałka $A : B $ oznacza $A$ implementuje/realizuje $B$.
\begin{itemize}
    \item $kCSD1d$ -  klasa bazowa, \\
    pola:
    \begin{itemize}
        \item $X : X$ -  tak mi się wydaje.
        \item $src : X_M$ - chyba jasne.
        \item $distTable : v_0 \in \V$ - to profil potencjału, a raczej jego połowa.
        \item $bPotMatrix : [\IV]_{nm} = v_{x_m}(y_n), y_n\in X^N, x_m\in X^M$
        \item $bSrcMatrix : [\IC]_{ml} = c_{x_m}(y_l), x_m\in X^M, y_l\in X$ 
        \item $KPot : K$ - tak to macierz jądra p-ni aproksymacyjnej.
        \item $interpCross :[\IC \IV] $ - to coś przenosi rzeczy z przestrzeni
        aproksymacyjnej do przestrzeni prądów.
        \item $csdEst: c \in \C$ - wynik.
    \end{itemize}
    metody:
    \begin{itemize}
        \item $create\_dist\_table() \to  Lc_0$ - tworzy $distTable$, czyli profil funkcj 
        bazowej z przestrzeni $\V (X_N,X_M,X) \mapsto v_0 \in \V$
        \item $bPotMatrixCalc \to (v_0, X_N,X_M,X)\mapsto \{c_x(y):x\in X_M, y\in X_N \}$ 
        \item $estimate$
        \item $chooseLambda$
        \item $calcCvError$
    \end{itemize}
    \item $kCSD1d\_ICA$ - klasa potomek, dodaje analiza składowych głównch
\end{itemize}

\section{Schemat działania}
Niech $d_{in}$ - dane wejściowe, $d_{out}$ wyjściowe, wtedy:
$d_{out}= (\IC \circ \IV \circ K^{-1} )( d_{in})$


%KROSWALIDACJA
\section{Rozszerzenie kroswalidacji}
Pytanie czy da się tę metodą dobrze rozszerzyć?
Pomysł prosty, zrobić modyfikacje w ten sposób:
Użytkownik podaje najbardziej prawdopodobne wartości, my zaś je wariujemy w
określonych granicach.


\section{Appendix matematyczny}
Rzecz warta rozkminy operator $K$.

%
% to co musze zmienić   
%
\end{document}

%TODO: trzeba zrobic porządek ze zmianą baz
